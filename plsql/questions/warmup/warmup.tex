\documentclass[12pt]{article}
\usepackage{fontspec}

\setmainfont[Mapping=tex-text]{Amiri}
\setsansfont{Arial}
\setmonofont[Mapping=tex-text]{Inconsolata}




%%%% Includes %%%%
\usepackage[margin=1in]{geometry}
\usepackage{titlesec}
\usepackage{hyperref}



%%%% Setup %%%%
\title{\rule{\linewidth}{0.5mm}\\
\textbf{Assignment 1:}\\
PL/SQL Warmup\\
\rule{\linewidth}{1.0mm}}
\author{Wolfgang Richter \textnormal{$\langle$}\href{mailto:wolfgang.richter@gmail.com}{wolfgang.richter@gmail.com}\textnormal{$\rangle$}}
\titleformat
    {\section}[hang]{\normalfont\Large\bfseries}{Problem \thesection. }{0pt}{}
\titleformat
    {\subsection}[hang]{\normalfont\bfseries}{Question \thesubsection. }{0pt}{}

%%%% Main document %%%%
\begin{document}

\maketitle

\section{Securely Storing Passwords}

Design a table that can store user IDs, usernames, and passwords.  User IDs
should be a unique numeric identifier.  No user should share the same user ID.
Usernames should be unique strings identifying each user.  Assume that their
maximum UTF-8 character length is 256 characters.  Assume also that passwords
are UTF-8 strings with a maximum length of 256 characters.  However, unlike
usernames, we should never store passwords in the clear.  Passwords are
sensitive pieces of information.  If a hacker broke into the database, they
would be able to steal people's passwords which are probably reused on other
websites.  Design some way of storing passwords which is secure and does not
store plaintext.

\subsection{Why is it a good idea to use UTF-8 for string input from users?}

\subsection{What are the options for securely storing passwords?}

\subsection{How would you characterize the performance tradeoffs between these
different options?  For example, could you design an experiment to benchmark
different strategies in PL/SQL?}

\subsection{Using the benchmark you designed, what is the empirical (implement
and run your benchmark) cost of adding complexity over simply storing passwords
in plaintext?}

\newpage\section{Many Different Index Flavors}

Tables in databases can easily grow to enormous size with hundreds of millions
of rows.  This causes tension with the database's primary mission: serve user
queries as quickly as possible.  As a DBA, the key tool generally applied to
speed up user queries is an index.  Generally, indexes create a summary with
``pointers" into a larger set of data.  For common user queries---those which
are repeated frequently---creating an index can greatly decrease the latency
users experience when running over very large tables.  But, there are many
different types of indexes which can be created.  Knowing when to use which
type is an important skill to develop.

\subsection{What are the types of indexes supported by Oracle DB today (don't
read further until you know the answer to this question)?}

\subsection{When would you use a B-tree style index?  When would you use a
bitmap index?}

\subsection{What is the difference between a local index and a global index?}

\subsection{Describe a scenario where a function-based index would be a good
choice.}

\subsection{Over what kind of data would you use a domain index?}

\subsection{When would you use a reverse key index?}

\subsection{What is index block contention? What type of query throughput does
it hurt?}

\end{document}
